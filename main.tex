\PassOptionsToPackage{hyphens}{url}
\documentclass[superscriptaddress, nofootinbib,  amsmath, amssymb, preprint]{revtex4-2}

\usepackage[margin=1in]{geometry} 
\usepackage[english]{babel}
\usepackage[utf8]{inputenc}
\usepackage[]{graphicx}
\usepackage{xspace}
\usepackage{siunitx}
\usepackage{mhchem}
\DeclareSIUnit\angstrom{\text {Å}}
\usepackage{orcidlink}
\usepackage{hyperref}
\usepackage{fontawesome5}
\usepackage{natmove}
\usepackage{placeins}

\usepackage{xparse}
\usepackage{hyperref}
\usepackage[a-1b]{pdfx}

\newcommand{\githublink}[2]{
  \href{https://github.com/#1/#2}{\faGithub\ \url{#1/#2}}
}

\newcommand{\twitterlink}[1]{
  \href{#1}{\faTwitter}
}

\newcommand{\zenodolink}[1]{
  \href{https://doi.org/#1}{\faArchive\ \url{#1}}
}

%https://twitter.com/SamCox822/status/1641484192566460416?s=20


\newcommand{\hflogo}{%

\includegraphics[height=.9em]{figures/huggingface.png}
}

\newcommand{\huggingfacelink}[2]{
  \href{https://huggingface.co/spaces/#1/#2}{\hflogo \url{#1/#2}}
}

\newcommand{\huggingfacehublink}[2]{
  \href{https://huggingface.co/#1/#2}{\hflogo \url{#1/#2}}
}

% Adjusted Hyperref Setup for Automatically Colored Text Links
\hypersetup{
  colorlinks=true,       % Enables colored links
  breaklinks=true,
  urlcolor=blue,         % Sets the color of URL links
  linkcolor=blue,        % Sets the color of internal links
  citecolor=blue,        % Sets the color of citation links
  filecolor=blue,        % Sets the color of file links
  allcolors=blue,        % Ensures all link types are blue by default
  pdftitle={Title},      % PDF Title
  pdfauthor={Author}     % PDF Author
}



% =====================================================
% packages for creating code listings 
\usepackage{listings, xcolor}
\definecolor{codegreen}{rgb}{0,0.6,0}
\definecolor{codegray}{rgb}{0.5,0.5,0.5}
\definecolor{codepurple}{rgb}{0.58,0,0.82}
\definecolor{tqblue}{HTML}{08293d}
\definecolor{backcolour}{HTML}{fefdf5}

\lstdefinestyle{pythonstyle}{
    backgroundcolor=\color{backcolour},   
    commentstyle=\color{codegreen},
    keywordstyle=\color{magenta},
    numberstyle=\tiny\color{codegray},
    stringstyle=\color{codepurple},
    basicstyle=\ttfamily\footnotesize\color{tqblue},
    breakatwhitespace=false,         
    breaklines=true,
    postbreak=\mbox{\textcolor{magenta}{$\hookrightarrow$}\space},                 
    captionpos=b,                    
    keepspaces=true,                 
    numbers=left,                    
    numbersep=5pt,                  
    showspaces=false,                
    showstringspaces=false,
    showtabs=false,                  
    tabsize=2
}

\lstset{style=pythonstyle}
\hbadness=99999 

\newcolumntype{C}{>{$}c<{$}}

\AtBeginDocument{%
  \heavyrulewidth=.08em
  \lightrulewidth=.05em
  \cmidrulewidth=.03em
  \belowrulesep=.65ex
  \belowbottomsep=0pt
  \aboverulesep=.4ex
  \abovetopsep=0pt
  \cmidrulesep=\doublerulesep
  \cmidrulekern=.5em
  \defaultaddspace=.5em
}



\usepackage[most]{tcolorbox}

\tcbset {
  base/.style={
    arc=0mm, 
    bottomtitle=0.5mm,
    boxrule=0mm,
    colbacktitle=black!10!white, 
    coltitle=black, 
    fonttitle=\bfseries, 
    left=2.5mm,
    leftrule=1mm,
    right=8.5mm,
    title={#1},
    toptitle=0.75mm, 
    width=\textwidth,
    breakable
  }
}

\definecolor{brandblue}{rgb}{0, 0.27843137254902, 0.466666666666667}
\newtcolorbox{agentinteraction}[1]{
  colframe=brandblue, 
  base={#1}
}


\definecolor{brandbred}{rgb}{0.63921568627451, 0, 0}
\newtcolorbox{agentinteraction2}[1]{
  colframe=brandbred, 
  base={#1}
}


\newtcolorbox{subbox}[1]{
  colframe=black!30!white,
  base={#1}
  }

\usepackage{csquotes}
\usepackage[acronym, nonumberlist]{glossaries}
\makeglossaries

\newacronym{llm}{LLM}{large language model}
\newacronym{gpt}{GPT}{generative pretrained transformer}
\newacronym{api}{API}{application programming interface}
\newacronym{ml}{ML}{machine learning}
\newacronym{lift}{LIFT}{language-interfaced fine-tuning}
\newacronym{icl}{ICL}{in-context learning}
\newacronym{peft}{PEFT}{parameter efficient fine-tuning}
\newacronym{lora}{LoRA}{low-rank adaptors}
\newacronym{gpr}{GPR}{Gaussian process regression}
\newacronym{ga}{GA}{genetic algorithm}
\newacronym{svm}{SVM}{support vector machine}
\newacronym{rf}{RF}{random forest}

\newacronym{ord}{ORD}{Open Reaction Database}
\newacronym{bo}{BO}{Bayesian optimization}
\newacronym{id}{ID}{inverse design}
\newacronym{mad}{MAD}{median absolute deviation}
\newacronym{eln}{ELN}{electronic lab notebook}
\newacronym[shortplural=LIMS]{lims}{LIMS}{laboratory information system}
\newacronym{ui}{UI}{user interface}
\newacronym{nlm}{NLM}{national library of medicine} 
\newacronym{dft}{DFT}{density functional theory}
\newacronym{cot}{COT}{chain of thought}
\newacronym{gui}{GUI}{graphical user interface}
\newacronym{pdb}{PDB}{protein data bank}
\newacronym{rlhf}{RLHF}{reinforcement learning from human feedback}
\newacronym{json}{JSON}{JavaScript object notation}
\newacronym{smiles}{SMILES}{simplified molecular-input line-entry system}
\newacronym{selfies}{SELFIES}{self-referencing embedded strings}
\newacronym{ai}{AI}{artificial intelligence}
\newacronym{nlp}{NLP}{natural language processing}
\newacronym{ner}{NER}{named entity recognition}
\newacronym{cas}{CAS}{Chemical Abstract Services}
\newacronym{mae}{MAE}{mean absolute error}
\newacronym{inchi}{InChI}{international chemical identifier}
\newacronym{mapi}{MAPI}{Materials Project \gls{api}}
\newacronym{rouge}{ROUGE}{Recall-Oriented Understudy for Gisting Evaluation}
\newacronym{html}{HTML}{HyperText Markup Language}
\newacronym{doi}{DOI}{digital object identifier}
\newacronym{ocr}{OCR}{optical character recognition}

\usepackage{tabularx} % For flexible tables with adjustable column widths
\usepackage{booktabs} % For better table lines (\toprule, \midrule, \bottomrule)
\usepackage{cleveref}

\let\originalcite\cite
\renewcommand{\cite}[1]{\unskip~\originalcite{#1}}

\usepackage{setspace}

\clubpenalty=10000
\widowpenalty=10000
\displaywidowpenalty=10000
\usepackage{titlesec}

\titlespacing{\subsection}
    {0pt}{9pt}{6pt}

\usepackage{array}
\usepackage{ragged2e}
\usepackage{nicefrac}
\usepackage[caption=false]{subfig}

\newcolumntype{P}[1]{>{\raggedright\arraybackslash}p{#1}}

\begin{document}

\title{Bayesian Optimization Hackathon for Chemistry and Materials}

% %ToDo: add contributed equally indication for participants

\newcommand{\equalcont}{\thanks{These authors contributed equally}}

\author{Kevin~Maik~Jablonka~\orcidlink{0000-0003-4894-4660}}
\email{mail@kjablonka.com}
\affiliation{Laboratory of Molecular Simulation (LSMO), Institut des Sciences et Ing\'{e}nierie Chimiques, Ecole Polytechnique F\'{e}d\'{e}rale de Lausanne (EPFL), Sion, Valais, Switzerland.}

\author{Qianxiang Ai~\orcidlink{0000-0002-5487-2539}}
\equalcont
\affiliation{Department of Chemical Engineering, Massachusetts Institute of Technology, Cambridge, Massachusetts 02139, United States.}

%\email{qai@mit.edu}


\author{Alexander~Al-Feghali~\orcidlink{0009-0004-8377-7049}} 
\equalcont
%\email{alexander.al-feghali@mail.mcgill.ca}
\affiliation{Department of Chemistry, McGill University, Montreal, Quebec, Canada.}

\author{Shruti~Badhwar~\orcidlink{0000-0002-3167-5348}}
\equalcont
\affiliation{Reincarnate Inc.}  
%\email{shruti@reincarnateartificial.com}

\author{Joshua~D.~Bocarsly~\orcidlink{0000-0002-7523-152X}}
\equalcont
 \affiliation{Yusuf Hamied Department of Chemistry, University of Cambridge, Lensfield Road, Cambridge, CB2 1EW, United Kingdom.} 
 
%\email{jb2382@cam.ac.uk} 

\author{Andres~M~Bran~\orcidlink{0000-0002-4432-3667}}
\equalcont
\affiliation{Laboratory of Artificial Chemical Intelligence (LIAC), Institut des Sciences et Ing\'{e}nierie Chimiques, Ecole Polytechnique F\'{e}d\'{e}rale de Lausanne (EPFL), Lausanne, Switzerland.} 
\affiliation{National Centre of Competence in Research (NCCR) Catalysis, Ecole Polytechnique F\'{e}d\'{e}rale de Lausanne (EPFL), Lausanne, Switzerland.}


\author{Stefan~Bringuier~\orcidlink{0000-0001-6753-1437}}
\equalcont
\affiliation{Independent Researcher, San Diego, CA, United States.}
%\email{stefanbringuier@gmail.com}

\author{L.~Catherine~Brinson~\orcidlink{0000-0003-2551-1563}}
\equalcont
\affiliation{Mechanical Engineering and Materials Science, Duke University, United States.}
%\email{cate.brinson@duke.edu}



\author{Defne~Circi~\orcidlink{0000-0002-5761-0198}}
\equalcont
\affiliation{Mechanical Engineering and Materials Science, Duke University, United States.}
%\email{defne.circi@duke.edu}

\author{Sam~Cox~\orcidlink{0000-0002-4441-9327}}
\equalcont
\affiliation{Department of Chemical Engineering, University of Rochester, United States.}
%\email{swrig30@ur.rochester.edu}

\author{Wibe~A.~de~Jong~\orcidlink{0000-0002-7114-8315}} 
\equalcont
\affiliation{Applied Mathematics and Computational Research Division, Lawrence Berkeley National Laboratory, Berkeley, CA 94720, United States.
}

\author{Matthew~L.~Evans~\orcidlink{0000-0002-1182-9098}}
\equalcont
 \affiliation{Institut de la Matière Condensée et des Nanosciences (IMCN), UCLouvain, Chemin des Étoiles 8, Louvain-la-Neuve, 1348, Belgium.}
 \affiliation{Matgenix SRL, 185 Rue Armand Bury, 6534 Gozée, Belgium.}
 %\email{matthew.evans@uclouvain.be}

 \author{Nicolas~Gastellu~\orcidlink{0000-0002-4052-076X}}
 \equalcont
\affiliation{Department of Chemistry, McGill University, Montreal, Quebec, Canada.}
%\email{nicolas.gastellu@mail.mcgill.ca} 

\author{Jerome~Genzling~\orcidlink{0009-0007-4728-1478}} 
\equalcont
\affiliation{Department of Chemistry, McGill University, Montreal, Quebec, Canada.}
%\email{jerome.genzling@mail.mcgill.ca}

\author{Mar\'ia~Victoria~Gil~\orcidlink{0000-0002-2258-3011}}
\equalcont
\affiliation{Instituto de Ciencia y Tecnolog\'ia del Carbono (INCAR), CSIC, Francisco Pintado Fe 26, 33011 Oviedo, Spain.}
%\email{victoria.gil@incar.csic.es}

\author{Ankur~K.~Gupta~\orcidlink{0000-0002-3128-9535}}
\equalcont
\affiliation{Applied Mathematics and Computational Research Division, Lawrence Berkeley National Laboratory, Berkeley, CA 94720, United States.
}


\author{Alishba~Imran}
\equalcont
\affiliation{Computer Science, University of California, Berkeley, Berkeley CA 94704, United States.}

\author{Sabine~Kruschwitz~\orcidlink{0000-0002-6296-4417}}
\equalcont
\affiliation{Bundesanstalt für Materialforschung und -prüfung, Unter den Eichen 87, 12205 Berlin, Germany.}
%\email{Sabine.Kruschwitz@bam.de}

\author{Anne~Labarre~\orcidlink{0000-0003-4939-3928}}
\equalcont
\affiliation{Department of Chemistry, McGill University, Montreal, Quebec, Canada.}
 %\email{Anne.Labarre@mail.mcgill.ca}  

\author{Jakub~Lála~\orcidlink{0000-0002-5424-5260}}
\equalcont
\affiliation{Francis Crick Institute, 1 Midland Rd, London NW1 1AT, United Kingdom.}
%\email{jakublala@gmail.com}
 
\author{Tao~Liu~\orcidlink{0000-0002-1082-5570}}
\equalcont
\affiliation{Department of Chemistry, McGill University, Montreal, Quebec, Canada.}
 %\email{tao.liu7@mail.mcgill.ca} 

\author{Steven~Ma~\orcidlink{0000-0006-9448-7332}}
\equalcont
\affiliation{Department of Chemistry, McGill University, Montreal, Quebec, Canada.}
 %\email{Steven.Ma@mail.mcgill.ca} 

\author{Sauradeep~Majumdar~\orcidlink{0000-0002-2095-3082}}
\equalcont
\affiliation{Laboratory of Molecular Simulation (LSMO), Institut des Sciences et Ing\'{e}nierie Chimiques, Ecole Polytechnique F\'{e}d\'{e}rale de Lausanne (EPFL), Sion, Valais, Switzerland.}
%\email{sauradeep.majumdar@epfl.ch}

\author{Garrett~W.~Merz~\orcidlink{0000-0003-4737-3931}}
\equalcont
\affiliation{American Family Insurance Data Science Institute, University of Wisconsin-Madison, Madison WI 53706, United States.}


\author{Nicolas~Moitessier~\orcidlink{0000-0001-6933-2079}}
\equalcont
\affiliation{Department of Chemistry, McGill University, Montreal, Quebec, Canada.}
%;  Department of Chemistry, McGill University, Montreal, Quebec, Canada; orcid.org/; 
%\email{nicolas.moitessier@mcgill.ca}

\author{Elias~Moubarak~\orcidlink{0000-0001-8271-6800}}
\equalcont
\affiliation{Laboratory of Molecular Simulation (LSMO), Institut des Sciences et Ing\'{e}nierie Chimiques, Ecole Polytechnique F\'{e}d\'{e}rale de Lausanne (EPFL), Sion, Valais, Switzerland.}
%\email{elias.moubarak@epfl.ch}



\author{Beatriz~Mouriño~\orcidlink{0000-0003-1670-3985}}
\equalcont
\affiliation{Laboratory of Molecular Simulation (LSMO), Institut des Sciences et Ing\'{e}nierie Chimiques, Ecole Polytechnique F\'{e}d\'{e}rale de Lausanne (EPFL), Sion, Valais, Switzerland.}
%\email{beatriz.buenomourino@epfl.ch}

\author{Brenden~Pelkie~\orcidlink{0000-0001-7638-6366}}
\equalcont
\affiliation{Department of Chemical Engineering, University of Washington, Seattle, WA 98105, United States.} 
%\email{bgpelkie@uw.edu} 

\author{Michael~Pieler~\orcidlink{0000-0001-9186-7045}}
\equalcont
\affiliation{OpenBioML.org}
\affiliation{Stability.AI}
%\email{ michael.pieler@gmail.com}

\author{Mayk~Caldas~Ramos~\orcidlink{0000-0001-5336-2847}}
\equalcont
\affiliation{Department of Chemical Engineering, University of Rochester, United States.}
%\email{mcaldasr@ur.rochester.edu}

\author{Bojana~Ranković~\orcidlink{0000-0002-1476-6686}}
\equalcont
\affiliation{Laboratory of Artificial Chemical Intelligence (LIAC), Institut des Sciences et Ing\'{e}nierie Chimiques, Ecole Polytechnique F\'{e}d\'{e}rale de Lausanne (EPFL), Lausanne, Switzerland.}
\affiliation{National Centre of Competence in Research (NCCR) Catalysis, Ecole Polytechnique F\'{e}d\'{e}rale de Lausanne (EPFL), Lausanne, Switzerland.}


\author{Jacob~N.~Sanders~\orcidlink{0000-0002-2196-4234}}
\equalcont
\affiliation{Department of Chemistry and Biochemistry, University of California, Los Angeles, CA 90095, United States.}
%\email{jacosand@gmail.com}




\author{Philippe~Schwaller~\orcidlink{0000-0003-3046-6576}}
\equalcont
\affiliation{Laboratory of Artificial Chemical Intelligence (LIAC), Institut des Sciences et Ing\'{e}nierie Chimiques, Ecole Polytechnique F\'{e}d\'{e}rale de Lausanne (EPFL), Lausanne, Switzerland.}
\affiliation{National Centre of Competence in Research (NCCR) Catalysis, Ecole Polytechnique F\'{e}d\'{e}rale de Lausanne (EPFL), Lausanne, Switzerland.}

 
\author{Marcus~Schwarting}
\equalcont
 \affiliation{Department of Computer Science, University of Chicago, Chicago IL 60490, United States.}
 %\email{meschw04@uchicago.edu}


\author{Jiale~Shi~\orcidlink{0000-0002-5447-3925}}
\equalcont
\affiliation{Department of Chemical Engineering, Massachusetts Institute of Technology, Cambridge, Massachusetts 02139, United States.}
 %\email{jialele@mit.edu}
 
 \author{Berend~Smit~\orcidlink{0000-0003-4653-8562}}
%\email{berend.smit@epfl.ch}
\equalcont
\affiliation{Laboratory of Molecular Simulation (LSMO), Institut des Sciences et Ing\'{e}nierie Chimiques, Ecole Polytechnique F\'{e}d\'{e}rale de Lausanne (EPFL), Sion, Valais, Switzerland.}

\author{Ben~E.~Smith~\orcidlink{0000-0001-9673-2449}}
\equalcont
 \affiliation{Yusuf Hamied Department of Chemistry, University of Cambridge, Lensfield Road, Cambridge, CB2 1EW, United Kingdom.} 
%\email{bs542@cam.ac.uk} 


\author{Joren~Van~Herck~\orcidlink{009-0005-5108-5061}}
\equalcont
\affiliation{Laboratory of Molecular Simulation (LSMO), Institut des Sciences et Ing\'{e}nierie Chimiques, Ecole Polytechnique F\'{e}d\'{e}rale de Lausanne (EPFL), Sion, Valais, Switzerland.}
%\email{joren.vanherck@epfl.ch}


\author{Sean~Warren~\orcidlink{0000-0002-3670-0354}}
\equalcont
%\affiliation{}
%\email{sean.warren@gatech.edu}



\author{Sylvester~Zhang}
\equalcont
\affiliation{Department of Chemistry, McGill University, Montreal, Quebec, Canada.}

\author{Xiaoqi~Zhang\orcidlink{0000-0002-6507-6490}}
\equalcont
\affiliation{Laboratory of Molecular Simulation (LSMO), Institut des Sciences et Ing\'{e}nierie Chimiques, Ecole Polytechnique F\'{e}d\'{e}rale de Lausanne (EPFL), Sion, Valais, Switzerland.}
%\email{xiaoqi.zhang@epfl.ch}


\author{Ghezal~Ahmad~Zia~\orcidlink{0000-0002-9082-9423}} 
\equalcont
\affiliation{Bundesanstalt für Materialforschung und -prüfung, Unter den Eichen 87, 12205 Berlin, Germany.}
%\email{Ghezal-Ahmad.Zia@bam.de} 



% organiziers 

\author{KJ~Schmidt}
%\email{kj.schmidt913@gmail.com}
\affiliation{Globus, University of Chicago, Data Science and Learning Division, Argonne National Lab, United States.}

\author{Ian~Foster~\orcidlink{0000-0003-2129-5269}}
\affiliation{Department of Computer Science, University of Chicago, Data Science and Learning Division, Argonne National Lab, United States.}

\author{Ben~Blaiszik~\orcidlink{0000-0002-5326-4902}}
\email{blaiszik@uchicago.edu}
\affiliation{Globus, University of Chicago, Data Science and Learning Division, Argonne National Lab, United States.}








\begin{abstract}
The Acceleration Consortium and Merck KGaA hosted a 2-day virtual hackathon on March 27-28, 2024, bringing together scientists to explore, collaborate, and innovate in the field of Bayesian optimization for the physical sciences. Participants were encouraged to select or develop Bayesian optimization algorithms, apply them to benchmarking tasks, design new benchmarks, create instructional tutorials, and describe real-world applications. With over 100 participants across 60 academic, industry, and government organizations located in 38 cities, 14 countries, and 4 continents, this was a truly global event. % https://chatgpt.com/share/f6cd733f-1126-4151-86c5-d4b59d158dc3
The outputs from this event, including developed algorithms, benchmarks, and tutorials, will serve as valuable resources for the research community, in addition to the new skills learned and connections formed. Released projects and general information are available at \url{https://ac-bo-hackathon.github.io/} and other locations linked from individual project pages. This event demonstrated the potential of community-driven research efforts to accelerate advances in Bayesian optimization in chemistry and materials science.
\end{abstract}

\maketitle


% ToDo: 
% emphasize tooling/constrained prompting/guidance 

\section{Introduction}

Bayesian optimization (BO) has emerged as a powerful tool in optimizing complex and expensive-to-evaluate functions, often outperforming traditional search methods in a variety of scientific domains such as optimizing composition and processing parameters to maximize alloy yield strength or identifying synthesis pathways that maximize efficacy of HIV drugs. The goal of the AC BO Hackathon was to leverage the expertise of a diverse, global community to advance the development and application of BO techniques for solving critical challenges in the physical sciences. In the words of Michelle Duke, the "Hackathon Queen":

\begin{quote}
A hackathon is a short competition where people work together in teams to solve problems and challenges by coming up with solutions and ideas.
\end{quote}

The hackathon also aimed to foster collaboration and knowledge sharing among participants from different backgrounds, including academia, national laboratories, government agencies, and private industry. The event attracted 120 active participants from 44 teams, representing 41 academic institutions, 12 national labs, and 9 companies. Likewise, the participants were located in 38 cities, 14 countries, and 4 continents. These teams engaged in a fast-paced, innovative environment to explore the frontiers of Bayesian optimization.

\subsection{Event Highlights}

The hackathon featured a series of collaborative tasks that encouraged participants to choose one or multiple of the following special topics:
\begin{itemize}
    \item \textbf{Select or develop BO algorithms} for specific scientific problems.
    \item \textbf{Apply BO algorithms} to newly designed benchmarking tasks.
    \item \textbf{Create instructional tutorials} to lower the barrier to entry for new users.
    \item \textbf{Describe real-world applications} where BO has the potential to make significant impacts.
\end{itemize}

Additionally, participants were welcome to submit projects in a "general" category.

Through these activities, participants not only honed their technical skills but also contributed to a collective effort to expand the toolkit available to researchers in the physical sciences. As a result, new connections were formed, new skills were acquired, and innovative ideas were inspired.

\subsection{Focus Areas and Outcomes}

The focus of the hackathon spanned a range of topics, including:
- **Tooling and constrained prompting:** Development of intuitive interfaces and guidance mechanisms for applying BO in practical settings.
- **Benchmark design:** Creation of new benchmarking datasets to evaluate and compare different BO algorithms.
- **Algorithm development:** Refinement and innovation of BO methodologies tailored to the unique challenges of materials science and chemistry.
- **Educational resources:** Generation of tutorials and case studies to support learning and adoption of BO techniques.

These outputs are available online and will continue to support the research community in pushing the boundaries of what is possible with Bayesian optimization.


We use Bayesian optimization (BO) because it often outperforms other search methods

Task: Find the set of values for x1 to x6 that minimizes a target property

Task: Find the composition/processing parameters that maximize alloy yield strength
Task: Find the synthesis pathway maximizes binding affinity to HIV protease


A hackathon is a short competition where people work together in teams to solve problems and challenges by coming up with solutions and ideas.

AC BO Hackathon: At-a-glance

https://ac-bo-hackathon.github.io/
44 Teams/projects
120 Active participants
41 Academic institutions
12 Nat’l labs/gov’t, etc.
9 Companies

Hosts: Acceleration Consortium, Merck KGaA

Preparation for the hackathon - 111 GitHub Classroom assignments accepted

\begin{figure}[h!]
    \centering
    % \captionsetup{justification=centering}
    \includegraphics[width=1\textwidth]{latex/figures/world_map.png}
    \caption{\textbf{Demographic distributions of the participating teams and their affiliations}. 
 \label{fig:map}}
\end{figure}



\begin{table*}[]
\caption{List of projects and project types, with links to corresponding website project pages, repositories, videos, and social media posts.}
\label{tab:projects}
\setlength{\extrarowheight}{0.4em}
\begin{tabularx}{\textwidth}{>{\centering\arraybackslash}p{1cm} X >{\centering\arraybackslash}X}
\toprule
\# & Project Name & Links \\ \midrule
\href{https://example.com}{\#1} & Project A & 
\href{https://github.com/example}{\faGithub} \,
\href{https://youtube.com}{\faVideo} \,
\href{https://twitter.com}{\faTwitter} \tabularnewline
\href{https://example.com}{\#2} & Project B & 
\href{https://github.com/example}{\faGithub} \,
\href{https://youtube.com}{\faVideo} \,
\href{https://linkedin.com}{\faLinkedin} \tabularnewline
\href{https://example.com}{\#3} & Project C & 
\href{https://github.com/example}{\faGithub} \,
\href{https://youtube.com}{\faVideo} \,
\href{https://twitter.com}{\faTwitter} \tabularnewline
\bottomrule
\end{tabularx}
\end{table*}







\section*{Acknowledgements}

%\printglossaries

% \bibliographystyle{achemso}
% \bibliography{refs}

\end{document}